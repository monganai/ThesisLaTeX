\chapter{Introduction}

\section{Motivation }

Personal experiences were the main driving factor in my motivation to pursue this study. As a mountain biker with 10 years of experience the author has sustained their fair share of minor injuries, but witnessing injuries sustained by more venerable fellow riders are sometimes more impactful. Last summer on a seemingly normal spin with a friend, we discovered a woman lying injured off on the trail side, incapacitated and unable to call for help so we did on her behalf. Multiple phone calls later to aid the first responders in locating us they arrived - around 1 hour after impact.  This experience made the author realize how useless your mobile phone is to you in these situations when one is unable to even pick it up. Non fatal injuries can easily become fatal, when one is alone, far from help and potentially incapacitated.





\section{Aims}


The  aim of this project was to develop an android application for real-time fall detection for cyclists,  automating the process of requesting assistance, and to reduce response time in the event of an accident. Before development of the application the author set himself strict aims to achieve. 

\subsection*{Simplistic and Intuiative}

After the initial set up process, to carry out the main use case: crash detection would be started and stopped with a single press of a button. Start the service, put your phone in your pocket and enjoy your time on you bike with piece of mind. Simple and convenient to use, removing the possibility of confusion for the end user, as the end users will be members of  the general public.
A simple user interface is important as the setting to which the app would be outdoors in potentially harsh weather conditions, external factors such as glare from the sun  and the possibility of moisture on the screen make high detailed, small user interface elements unsuitable. Less is more in this scenario.


\subsection*{Diverse}

Many existing systems are discipline specific, only working for one aspect of cycling i.e., for cross country usage only. The author intends this app to have the potential to work for all disciplines of cycling. Targeting single disciplines would drastically reduce the number of potential users as well as producing highly undesirable, inaccurate results if used for the incorrect discipline.

\subsection*{Standalone}

Utilizing android smartphones built in sensors removes the need for extraneous external equipment for ride monitoring.  The author intends the app to work as expected with one's phone placed in their pocket or bag, requiring no extra mounting equipment for either the rider or the bike.


\subsection*{Enjoyable user experience}
Many existing solutions exhibit deal breaking issues which ultimately causes the end user to stop using the system, the author aimed to eradicate the pitfalls present in other systems leading to a better user experience. 


\subsection*{Efficiency}
Performance in terms of battery usage is of utmost importance, heavy battery usage would have the potential to kill the phone when one would need it most - in an emergency. Every possible optimization in terms of battery will be made where possible - without impacting performance.

\section{Personal Goals}
In addition to the aims of this project the author had set some personal goals to achieve from undertaking this project.


\subsection*{Develop a fully functional application}
Having had brief experience working with android studio before undertaking this project to develop simple applications, most of which were interfaces for arduino circuits connected via bluetooth, the author had never developed such a large scale complex application prior to this project. The author was excited to broaden his skill set and develop an application ready to be published to the google play store. 

\subsection*{Work with Embedded sensors}
Having experience working with microcontrollers and various sensors, the author was excited to utilize the plethora of available sensors present in android smartphones today. An investigation will be conducted to identify the most suitable sensors to utilize for this project.



\subsection*{Collect and Analyse Real World Data}
Datasets for what a bike accident looks like in terms of sensor values are few and far between, the author was excited to conduct their own research with many unknowns to which would need to be investigated and discovered. Having very few similar documented studies available they were very interested to study this particular system in the domain of cycling.  


\subsection*{Real world testing}
The author was aware before undertaking this study that it would involve a lot of real world data collection, analysis and testing. Being a crash detection application testing could not be simulated sitting at a desk, which meant all testing would need to be conducted in the real world which proved both challenging and exciting.

\newpage

\section{Readers Guide}

\subsection* {2 - Background}
This section will discuss the concept of fall/accident detection, exploring both the existing cycling and medical applications. The main approaches to the problem of fall detection will be discussed and the pros and cons of each type of system will be discussed.


\subsection* {3 - Design}
Chapter 3 will discuss the design of RideSafe in terms platform choices,  User interface design as well as the design of the systems underlying architecture design.


\subsection* {4 - Implementation}
Chapter 4 will discuss the methods which were implemented in the development of RideSafe,on a per activity basis. A detailed explanation of the methodologies used to collect crash data and train the applications model will also be contained in this chapter. 


\subsection* {5 - Evaluation}


Chapter 5 is a discussion of results from various tests RideSafe was subjected to. Results from both real world on location testing as well as CarPark tests will be displayed and analyzed. Performance metrics for different aspects of RideSafe will be discussed. 



\subsection* {6 - Conclusion}

In Chapter 6 the author will discuss potential future uses for RideSafe as well as features which could be added and improved with further work. Conclusions from the the project itself will also be discussed.



